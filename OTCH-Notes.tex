\documentclass[letterpaper]{article}
\usepackage[margin=1in,letterpaper]{geometry}
\usepackage[utf8]{inputenc}
\usepackage[english]{babel}
\usepackage{amsfonts}
\usepackage{amsthm}
\usepackage{amsmath}
\usepackage{amssymb}
\usepackage{mathtools}
\usepackage{tikz-cd}
\usepackage{color}
\usepackage{outlines}

\synctex=1

\newcommand{\Sp}{\mathbf{Sp}}
\newcommand{\Fun}{\mathbf{Fun}}
\newcommand{\Aeff}{\mathbf{A}^\textnormal{eff}}
\newcommand{\finset}{\mathbf{Fin}}
\newcommand{\calF}{\mathcal{F}}
\newcommand{\Spc}{\mathbf{Spc}}
\newcommand{\Orb}{\mathbf{O}}
\newcommand{\bbT}{\mathbb{T}}
\newcommand{\MapSp}{\textnormal{MapSp}}
\newcommand{\op}{\textnormal{op}}
\newcommand{\THH}{\textnormal{THH}}
\newcommand{\CycSp}{\mathbf{CycSp}}
\newcommand{\LEq}{\textnormal{LEq}}
\newcommand{\id}{\textnormal{id}}
\newcommand{\TC}{\textnormal{TC}}
\newcommand{\Alg}{\mathbf{Alg}}
\newcommand{\bbE}{\mathbb{E}}
\newcommand{\Ex}{\textnormal{Ex}}
\newcommand{\Map}{\mathbf{Map}}
\newcommand{\lex}{\textnormal{lex}}
\newcommand{\calD}{\mathcal{D}}
\newcommand{\bbZ}{\mathbb{Z}}
\newcommand{\res}{\textnormal{res}}
\newcommand{\ind}{\textnormal{ind}}
\newcommand{\free}{\textnormal{free}}


\theoremstyle{definition}
\newtheorem{definition}{Definition}[section]
\newtheorem{ex}{Example}
\newtheorem{lemma}{Lemma}
\newtheorem*{goal}{Goal}

\theoremstyle{remark}
\newtheorem{remark}{Remark}
\newtheorem{Q}{Question}

\theoremstyle{plain}
\newtheorem{prop}{Proposition}
\newtheorem{theorem}{Theorem}
\newtheorem{cor}{Corollary}

\setcounter{tocdepth}{3}
\setcounter{secnumdepth}{3}

\begin{document}

\title{Notes - On Topological Cyclic Homology}
\author{Johnson Tan}
\date{\today}
\maketitle

\tableofcontents

\section{Cyclotomic Spectra}

\subsection{Cyclotomic spectra and TC}

\begin{definition}
		A cyclotomic spectrum is a spectrum $X$ with a $\bbT$-action together with $\bbT$-equivariant maps $\varphi_p\colon X\rightarrow X^{tC_p}$. For $p$ a fixed prime, a $p$-cyclotomic spectrum is a spectrum $X$ with a $C_{p^\infty}$-action and a $C_{p^\infty}$-equivariant map $\varphi_p\colon X\rightarrow X^{tC_p}$.
\end{definition}

\begin{remark}
		Here we are using the identification $\bbT\simeq\bbT/C_p$ (resp. $C_{p^\infty}\simeq C_{p^\infty}/C_p$) via the map $x\mapsto[x^{1/p}]$ to identify the residual $\bbT/C_p$-action (resp. $C_{p^\infty}/C_p$-action) of $X^{tC_p}$.
\end{remark}

\begin{ex}
		Let $S^0\in\Sp^{B\bbT}$ with trivial $\bbT$-action and $f\colon B\bbT\rightarrow B(\bbT/C_p)$. Then we the following composition of maps
$$S^0\rightarrow f_*f^*(S^0)\rightarrow(f^*(S^0))^{tC_p}$$
		in $\Sp^{B(\bbT/C_p)}\simeq\Sp^{B\bbT}$. It will later be shown that as a cyclotomic spectra, $S^0\simeq\THH(S^0)$.
\end{ex}

\begin{remark}
		In the notes they worried about the equivariant equivalence of the map a little more. Not quite sure why they do this...
\end{remark}

\begin{definition}
		The $\infty$-category of cyclotomic spectra, denoted $\CycSp$, is defined as the lax equalizer
		$$\CycSp := \LEq((\id)_p,((-)^{tC_p})_p\colon\Sp^{B\bbT}\rightrightarrows\prod_p\Sp^{B\bbT}).$$
		Similarly for a fixed prime $p$ the $\infty$-category of $p$-cyclotomic spectra, denoted $\CycSp_p$, is defined as the lax equalizer
		$$\CycSp_p := \LEq(\id,(-)^{tC_p}\colon\Sp^{BC_{p^\infty}}\rightrightarrows\Sp^{BC_{p^\infty}}).$$
\end{definition}

Both $\CycSp$ and $\CycSp_p$ are presentable stable $\infty$-categories. The forgetful functors $\CycSp\rightarrow\Sp$ and $\CycSp_p\rightarrow\Sp$ reflect equivalences, are exact, and preserve all small colimits.

\begin{definition}
		Let $(X,(\varphi_p)_p)\in\CycSp$ (resp. $(X,\varphi_p)\in\CycSp_p$) be a cyclotomic (resp. $p$-cyclotomic) spectrum. The integral (resp. $p$-typical) topological cyclic homology, denoted $\TC(X)$ (resp. $\TC(X,p)$) is the mapping spectrum 
		$$\MapSp_\CycSp(S^0,X)(resp. \MapSp_{\CycSp_p}(S^0,X)).$$
		Let $R\in\Alg_{\bbE_1}(\Sp)$, then $\TC(R):=\TC(\THH(R))$ and $\TC(R,p):=\TC(\THH(R),p)$.
\end{definition}

\begin{remark}
		It will be shown later that $\THH(R)$ has a canonical structure of a cyclotomic spectrum.
\end{remark}

\begin{prop}
		Let $(X,(\varphi_p)_p)$. There is a functorial fiber sequence
		$$\TC(X)\rightarrow X^{h\bbT}\xrightarrow{(\varphi_p^{h\bbT}-\textnormal{can})_p}\prod_p(X^{tC_p})^{h\bbT}$$
		where $\textnormal{can}\colon X^{h\bbT}\simeq(X^{hC_p})^{h(\bbT/C_p)}\simeq(X^{hC_p})^{h\bbT}\rightarrow(X^{tC_p})^{h\bbT}$.
\end{prop}

\section{Topological Hochschild Homology}

\subsection{The Tate Diagonal}

\begin{goal}
		Produce a substitute for the diagonal map of spaces in the category of spectra.
\end{goal}

\begin{prop}
		Let $p$ be a prime. The functor $T_p\colon\Sp\rightarrow\Sp$ given by $X\mapsto(X^{\otimes p})^{tC_p}$ is exact, where $X^{\otimes p}$ denotes the $p$-fold self tensor product with the $C_p$-action given by the cyclic permutation of the factors.
\end{prop}

\begin{proof}
		
\end{proof}

Recall that there is an equivalence $\Fun^\Ex(\Sp,\Sp)\simeq\Fun^\lex(\Sp,\Spc)$ given by composing with $\Sigma^\infty$. Since $\id_\Sp$ corresponds to $\Omega^\infty$ under this correspondence, by the Yoneda lemma, we have an equivalence between the space of natural transformations $\Map_{\Fun^\Ex(\Sp,\Sp)}(\id_\Sp,F)$ and the mapping space $\Map_\Sp(S^0,F(S^0))$.

\begin{definition}
		The Tate diagonal is the natural transformation
		$$\Delta_p\colon\id_\Sp\rightarrow T_p\colon X\rightarrow(X^{\otimes p})^{tC_p}$$
		corresponding to the map $S^0\rightarrow(S^0)^{tC_p}$ associated to the free cyclotomic structure of $S^0$.
\end{definition}

\begin{theorem}
		Let $X\in\Sp$ be bounded below. Then the map $\Delta_p$ exhibits $(X^{\otimes p})^{tC_p}$ as the $p$-completion of $X$.
\end{theorem}

\begin{remark}
		Note that the existence of $\Delta_p$ uses the universal characterization of $\Sp$ as the stabilization of $\Spc$. In particular there is no lax symmetric monoidal transformation $C\rightarrow(C^{\otimes p})^{tC_p}$ as functors $\calD(\bbZ)\rightarrow\calD(\bbZ)$.
\end{remark}

\appendix

\section{Equivariant Stable Homotopy Theory}

\begin{goal}
		Reproduce as much of the required theory regarding spectra with $G$-action and genuine $G$-spectra with spectral mackey functors. In particular:
		\begin{itemize}
				\item All the different kinds of fixed points in $G\Sp$.
				\item Describe the fixed point functors for $C_{p^\infty}\Sp:=\lim_n C_{p^n}\Sp$ via forgetful maps.
				\item Similarly for $\bbT\Sp_\calF$ and furthermore achieve this using cyclonic spectra.
		\end{itemize}
\end{goal}

\subsection{Genuine $G$-Spectra}

\begin{definition}
		The $\infty$-category of genuine $G$-spectra is $G\Sp := \Fun^\oplus(\Aeff(\finset_G),\Sp)$. For $H\leq G$ and $X\in G\Sp$, the genuine $H$-fixed point spectrum of $X$ is defined as $X^H:=X(G/H)\simeq\MapSp(\Sigma^\infty_+ G/H, X)$.
\end{definition}

\begin{prop}
		The $\infty$-category $G\Sp$ is stable with $t$-structure where $X\in G\Sp$ is connective if it is pointwise connective. In particular $G\Sp^\heartsuit$ is equivalent to the classical category of mackey functors. There is an adjunction 
		$$\Sigma^\infty\dashv\Omega^\infty\colon G\Spc_*\leftrightarrows G\Sp$$
		where $\Omega^\infty(E)=\Omega^\infty\circ E\circ j$ and $j\colon\Orb^\op_G\rightarrow\Aeff(\finset_G)$ is the obvious inclusion. For $I\in\finset_G$, the suspension spectra $\Sigma^\infty_+ I$ is self dual in $G\Sp$.
\end{prop}

\begin{remark}
		In [NS] they define equivalences of $G$-spectra using geometrix fixed points. This is equivalent to equivalences detected by genuine fixed points.
\end{remark}

\subsubsection*{Todo}

\begin{outline}
		\1 Add in detail about $\Aeff$ and its self duality
\end{outline}

\subsection{Restriction, Induction, and the Wirthm\"uller Isomorphism}

Let $H$ be a subgroup of $G$, we will construct an ambidextrous adjunction
$$\res^G_H\colon G\Sp\leftrightarrows H\Sp\colon\ind^G_H.$$
To construct the above adjunction, we define adjunctions at the following steps:
\begin{itemize}
		\item Finite $G$ sets $\finset_G$.
		\item Span categories $\Aeff(\finset_G)$.
		\item Spectral Mackey Functors $\Fun^\oplus(\Aeff(\finset_G),\Sp)$.
\end{itemize}

Let $\res^G_H\colon\finset_G\rightarrow\finset_H$ denote the forgetful functor. Then the left adjoint is denoted by $\ind^G_H\colon\finset_H\rightarrow\finset_G$ which sends a finite $H$-set $X$ to the $G$-set $G\times_H X:= (G\times X)/H$ where $H$ acts by $h(g,x)=(gh^{-1},hx)$. In particular, $\ind^G_H$ sends $H/K$ to $G/K$. Since both $\res^G_H$ and $\ind^G_H$ preserve pullbacks both extend to functors $\res^G_H\colon\Aeff(\finset_G)\rightarrow\Aeff(\finset_H)$ and $\ind^G_H\colon\Aeff(\finset_H)\rightarrow\Aeff(\finset_G)$. Furthermore they are still adjoints at this level, i.e. $\res^G_H\dashv\ind^G_H$.

\begin{remark}
		Adjointness is nontrivial and does not follow immediately from $\Aeff$ being functorial.
\end{remark}

In fact by self duality of $\Aeff(\finset_G)$, we have that $\ind^G_H\dashv\res^G_H$. Finally we define $\res^G_H\colon G\Sp\rightarrow H\Sp$ by precomposition with $\ind^G_H\colon\Aeff(\finset_H)\rightarrow\Aeff(\finset_G)$ and similarly for $\ind^G_H$.	

\begin{remark}
		Observe that for $X\in G\Sp$ and $K\leq H\leq G$, we have that
		\begin{align*}
				(\res^G_H X)^K &= (\res^G_H X)(H/K)\\
							&= X(\ind^G_H(H/K))\\
							&= X(G/K)\\
							&= X^K.
		\end{align*}
		That is the genuine $K$-fixed points for $K\leq H$ do not change.
\end{remark}

\subsubsection*{Todo}
\begin{outline}
		\1 Why do $\ind,\res$ preserve pullbacks?
		\1 Lol if deranged enough then explaning why adjoints at the level of $\Aeff$.
\end{outline}

\subsection{Homotopy Fixed Points}

Recall that for a spectra with $G$-action $X\in\Sp^{BG}:=\Fun(BG,\Sp)$, the homotopy fixed points of $X$ is $X^{hG}:=\lim_{BG}X$. For a genuine $G$-spectra forgetting to the underlying spectra with $G$-action is encoded by the functor $u_G\colon G\Sp\rightarrow\Fun(BG,\Sp)$ given by restriction along $BG\hookrightarrow\Aeff(\finset_G)$. Then the homotopy $G$-fixed point functors are given by forgetting to $\Sp^{BG}$ then taking the usual homotopy fixed points. We can similarly define the homotopy $H$-fixed point functors by taking a further restriction to $\Sp^{BH}$ or equivalently by initially applying $\res^G_H$ to land in $H\Sp$.

By a result of Saul Glasman we have an alternate way of defining the homotopy fixed points. Let $\finset_G^\free\subset\finset_G$ denote the full subcategory of finite free $G$-sets.

\begin{theorem}
		The functor $G\Sp^\free:=\Fun^\oplus(\Aeff(\finset_G^\free),\Sp)\rightarrow\Fun(BG,\Sp)$ given by precomposition with the inclusion induces an equivalence of $\infty$-categories.
\end{theorem}

\begin{remark}
		Here the inclusion is given by sending $*\in BG$ to $G/e\in\Orb_G\subset\finset_G^\free\subset\Aeff(\finset_G^\free)$. In particular, $\Aeff(\finset_G^\free)$ is the free semiadditive $\infty$-category generated by $BG$.
\end{remark}

\begin{proof}
		Can be proved via Lawvere theory, we refer to notes by Maxime Ramzi.
\end{proof}

On the other hand we may also consider the restriction map $u_G\colon G\Sp\rightarrow G\Sp^\free$ induced by the inclusion $\finset_G^\free\subset\finset_G$. It is not hard to see that composing $u_G$ with the equivalence from before is the same as directly restricting to $BG$. Define the Borel completion which we denote by $b_G$ as the right adjoint to $u_G$.

\begin{remark}
		Abusing notation and conflating $G\Sp^\free$ with $\Sp^{BG}$, we may define the homotopy $H$-fixed point functor as $X\mapsto (u_G X)^{hH}$. Equivalently we first restrict to $H$-spectra than apply the corresponding functor, i.e. $X\mapsto(u_H\res^G_H X)^{hH}$.
\end{remark}

\begin{theorem}
		For $X\in G\Sp^\free$, there is an equivalence $X^{hH}\simeq(b_G X)^H$ for all $H\leq G$.
\end{theorem}

\begin{proof}
		First suppose that this is true for $H=G$. As shown above both adjoints $u_G$ and $b_G$ commute with restrictions. Then 
		$$(b_G X)^{hH} = (\res^G_H u_G b_G X)^{hH}\simeq(u_H b_H\res^G_H X)^{hH}\simeq(\res^G_H X)^{hH}\simeq(b_H\res^G_H X)^H\simeq(\res^G_H b_G X)^H\simeq(b_G X)^H.$$
		Thus it remains to prove the case $H=G$.
\end{proof}

\begin{theorem}
		The functor $b_G$ is fully faithful and the essential image consists of the full subcategory of $X\in G\Sp$ such that the natural map $X^H\rightarrow X^{hH}$ is an equivalence for all $H\leq G$ which we refer to as Borel-complete $G$-spectra.
\end{theorem}

\subsubsection*{Todo}

\begin{outline}
\1 Give proof for homotopy fixed points = genuine fixed points.
\1 Replace the above remark with proof that restriction and adjoints commute.
\1 Why is it true that $(u_H b_H\res^G_H X)^{hH}\simeq(\res^G_H X)^{hH}$?
\end{outline}


\subsection{Geometric Fixed Points}

We now construct the geometrix fixed point functors. Let $\calF$ be a family of subgroups closed under conjugation and subgroups. Then define

\[
(E\calF)^K
\begin{cases}
		* & k\in\calF\\
		\varnothing & k\not\in\calF
\end{cases}
\in G\Spc.
\]

\begin{ex}
		Fix a subgroup $H\subseteq G$, then define $\calF_H$ to be the collection of subgroups which do not contain a conjugate of $H$ as a subgroup.
\end{ex}

\begin{remark}
		We could alternatively describe such a collection of subgroups as a sieve in $\Orb_G$. In particular, the above example can be described as the maximal sieve not containing $G/H$.
\end{remark}

We will also be interested in the unreduced suspension of $E\calF$. That is
$$E\calF_*\rightarrow S^0\rightarrow \widetilde{E\calF},$$
in particular
\[
(\widetilde{E\calF})^K =
\begin{cases}
		*	&	K\in\calF\\
		S^0	&	K\not\in\calF.
\end{cases}
\]

In the case $\calF$ consists of only the trivial subgroup, then $E\calF = EG$ and $\widetilde{E\calF}=\widetilde{EG}$. 

\begin{lemma}
		Let $E\in G\Spc$, then $(E\wedge\Sigma^\infty\widetilde{E\calF})^K \simeq 0$ if $K\in\calF$ and $E^K\rightarrow(E\wedge\Sigma^\infty\widetilde{E\calF})^K$ is an equivalence if $K\not\in\calF$.  
\end{lemma}

\begin{remark}
		The equivalences in the lemma above are induced by the map $S^0\rightarrow\widetilde{E\calF}$ from the cofiber sequence.
\end{remark}

\begin{theorem}
		There is a smashing localization such that the local objects are $G$-spectra concentrated away from $\calF$, i.e. $E\rightarrow E\wedge\Sigma^\infty\widetilde{E\calF}$ is an equivalence.
\end{theorem}

\begin{remark}
		This is immediate from the lemma and that $\widetilde{E\calF}$ is idempotent from looking at the fixed points.
\end{remark}

Let $\finset_{\calF^c}\subseteq\finset_G$ denote the full subcategory of those fintie $G$-sets whose stabilizers are not in $\calF$.

\begin{theorem}
		The category of $G$-spectra concentrated away from $\calF$ is equivalent to $\Fun^\oplus(\Aeff(\finset_{\calF^c}),\Sp)$ and the inclusion of the precomposition with the map $\Psi\colon\Aeff(\finset_G)\rightarrow\Aeff(\finset_{\calF^c})$ that sends every $I\in\finset_G$ to the subset of points with stabilizers in $\calF$.
\end{theorem}

\begin{remark}
		Fix $H\leq G$ and consider $\calF$ as in the previous example. Observe that $K\not\in\calF$ if and only if there is a map of $G$-sets $G/H\rightarrow G/K$. In the case that $H$ is a normal subgroup of $G$, then $\finset_{\calF^c}$ is the category of finite $(G/H)-sets$. Indeed $\finset_{\calF^c}$ would be all finite $G$-sets whose stabilizers are subgroups of $H$.
\end{remark}

In particular, the localization functor is given by left Kan extension along $\Psi$, i.e.
$$\Psi_!\colon G\Sp\rightarrow\Fun^\oplus(\Aeff(\finset_{\calF^c}),\Sp).$$
We call $\Phi_!$ the geometric $H$-fixed point functor and denote it by $\Phi^H$ or $(-)^{\Phi H}$. In the case $H$ is normal, then this functor takes values in $(G/H)$-spectra.

\begin{remark}
		The geometric $H$-fixed point functors are symmetric monoidal.
\end{remark}

\subsection{Comparison of Fixed Points}

\subsection{$G = C_{p^\infty}$ or $\bbT$}

\subsubsection*{Todo}

\begin{outline}
\1 Specify a bit more about genuine fixed points
	\2 In particular where the natural transformation $(-)^H\rightarrow(-)^{\Phi H}$ comes from.
	\2 Also the composability of geometric fixed points, i.e. something like $\Phi^{H'/H}\circ \Phi^H\rightarrow\Phi^{H'}$ is an equivalence.
	\2 Mention how the geometric fixed points form a localization and what the corresponding reflective subcategory is.
\1 Do the same for homotopy fixed points
	\2 In particular how there is NO DIRECT relation with geometric fixed points.
\1 Write about the $C_{p^\infty}$ and $\bbT$ equivariant variants of the theory
	\2 In particular what results hold "essentially" analogously.
\1 Also specify how we can get borel $G$-spectra using spectral mackey functors.
\end{outline}


\end{document}
